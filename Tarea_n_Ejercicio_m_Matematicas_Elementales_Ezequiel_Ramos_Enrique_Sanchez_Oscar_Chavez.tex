\documentclass{article}
\usepackage[spanish]{babel}
\usepackage[utf8]{inputenc}
\usepackage{enumitem}
\usepackage{amsmath}
\usepackage{amssymb}
\usepackage{amsthm}
\usepackage{geometry}
\usepackage{setspace}
\usepackage{fancyhdr}
\geometry{letterpaper, margin=.7in}

% Edita rápido el ejercicio y la tarea
\newcommand{\ntarea}{6}
\newcommand{\nejercicio}{1}

\begin{document}

\pagestyle{fancy}
\lhead{Ejercicio \nejercicio}
\chead{Matemáticas Elementales}
\rhead{Tarea \ntarea}

\thispagestyle{empty}
\begin{large}
    \noindent Tarea \ntarea, Matem\'aticas Elementales \\ 
    Ejercicio \nejercicio \\
\end{large}
\rule{180mm}{0.4mm} \\
Jesús Ezequiel Ramos Moreno. Licenciatura en Computación Matemática\\
José Enrique Sánchez Martínez. Licenciatura en Matemáticas\\
Óscar Chávez. Licenciatura en Computación Matemática\\
Héctor Javier Sánchez Salazar. Licenciatura en Computación Matemática\\
\rule{180mm}{0.4mm} \\
\textbf{Ejercicio \nejercicio.}\\
% Enunciado del ejercicio
Enunciado del ejercicio\\
\rule{180mm}{0.4mm}\\ 
% Nombre del autor
Ejercicio redactado por Jesús Ezequiel Ramos Moreno. 

\end{document}